\documentclass[11pt,a4paper]{article}
\usepackage[utf8]{inputenc}
\usepackage[catalan]{babel}
\usepackage{amsmath}
\usepackage{amsfonts}
\usepackage{enumerate}
\usepackage{enumitem}
\usepackage{amssymb}
\usepackage{graphicx}
\usepackage{fancyhdr}
\usepackage{appendix}
\usepackage[hidelinks]{hyperref}
\usepackage{hyperref}
\usepackage{subfig}
\usepackage[ampersand]{easylist}
\usepackage{multirow}
\usepackage{amsmath}
\usepackage{amsfonts}
\newcommand\tab[1][1cm]{\hspace*{#1}}
\usepackage[left=2cm,right=2cm,top=2cm,bottom=2cm]{geometry} 
\renewcommand{\appendixname}{Webgrafia}
\renewcommand{\appendixtocname}{Webgrafia}
\renewcommand{\appendixpagename}{Webgrafia}


\begin{document}
\begin{titlepage}

\begin{flushleft}
Escola Politècnica Superior\\
\vspace*{0.15in}
Grau en Enginyeria Informàtica\\
\vspace*{0.15in}
Automatic Reasoning and Learning
\end{flushleft}

\begin{center}
\vspace{2.0cm}
\includegraphics[scale=0.3]{Figures/M-UdL.jpg} 
\vspace{2.0cm}

\begin{LARGE}
\textbf{Pràctica de Models Probabilístics:}\\ 
\vspace*{0.15in}
Indentificació de classes de vidres en l'escena del crim
\end{LARGE}
\vspace{1.0cm}

\begin{large}
\textbf{Data}: 25 de Juny, de 2018 \\
\textbf{Professor}: Ramón Béjar Torres \\
\begin{tabular}{ll}
\textbf{Nom: }Marc Melis Batalla & \textbf{DNI: }48257130W \\
\textbf{Nom: }Roger Truchero Visa  & \textbf{DNI: }48056539V \\
\end{tabular}
\end{large}

\vspace*{0.2in}
\begin{center}
\rule{120mm}{0.1mm}\\
\end{center}
\end{center}
\vspace*{0.2in}
\part*{Introducció}
L'objectiu del treball és realitzar aprenentatge basat en xarxes bayesianes, en lloc de basat en sistemes
de regles. Es tracta del problema de classificació de diferents classes de vidres en funció de diferents
propietats físiques dels mateixos. Aquest és el valor que té cada un dels 11 atributs que trobareu
per cada instància (propietats d'un cristall correctament identificat) en el fitxer glass.data que es troba juntament amb la pràctica.
\end{titlepage}

\lhead[\thepage]{\includegraphics[scale=0.05]{Figures/M-UdL.jpg}  }
\chead[]{\textbf{Pràctica de Models Probabilístics}}
\rhead[]{ARA}
\renewcommand{\headrulewidth}{0.5pt}
\renewcommand{\footrulewidth}{0.5pt}
\fancypagestyle{plain}{
\fancyhead[L]{}
\fancyhead[C]{}
\fancyhead[R]{\thepage}
\fancyfoot[L]{}
\fancyfoot[C]{}
\fancyfoot[R]{}
\renewcommand{\headrulewidth}{0pt}
\renewcommand{\footrulewidth}{0pt}
}
\pagestyle{fancy}
\vspace*{0.05in}

\tableofcontents

\vspace*{0.2in}

\listoffigures

\newpage

\part{Parseig de dades}

\newpage

\part{Aprenentatge de models amb K2}
Abans de començar a apendre els models, haurem de carregar el fitxer \textit{.arff} que conté les dades dels materials del crim.\\
\begin{figure}[hbtp]
\centering
\includegraphics[scale=0.4]{Figures/1.png}
\caption{Selecció del fitxer de dades }
\end{figure}
\\\\
Seguidament seleccionar el classificador BayesNet:\\
\begin{figure}[hbtp]
\centering
\includegraphics[scale=0.5]{Figures/2.png}
\caption{Classificador BayesNet}
\end{figure}
\\\\
I finalment, seleccionar l'algorisme d'aprenentatge K2 per, posteriorment, modificar-ne els seus paràmetres d'execució i poder obtenir diferents tipus de models:\\
\begin{figure}[hbtp]
\centering
\includegraphics[scale=0.38]{Figures/3.png}
\caption{Algorisme de búsqueda K2}
\end{figure}

\newpage

\section{Model A}
\textit{\textbf{Xarxes bayesianes obtingudes amb K2 a partir d'un model inicial buit (sense arestes inicials) i amb un ordre entre les variables escollit a l'atzar i amb un valor determinat pel nombre màxim de pares per variable (paràmetre O en l'algoritme K2).}}\\
\begin{itemize}
\item Ajustem els paràmetres del K2 per satisfer les següents condicions:
	\begin{itemize}
	\item \textbf{Model inicial sense arestes}: \textit{initAsNaiveBayes} $\rightarrow$ \textit{False}
	\item \textbf{Ordre aleatori de selecció de variables}: \textit{randomOrder} $\rightarrow$ \textit{True}
	\item \textbf{Màxim nombre de pares per variable}: \textit{maxNrOfParents} $\rightarrow$ \textit{N, on N $\epsilon$  $\mathbb{N} - \{0\}$ }
	\end{itemize}
	\begin{figure}[hbtp]
	\centering
	\includegraphics[scale=0.4]{Figures/5.png}
	\caption{Paràmetres K2 model A}
	\end{figure}
\item Visualitzem el graf generat en l'execució:
	\begin{figure}[hbtp]
	\centering
	\includegraphics[scale=0.4]{Figures/r1.png}
	\caption{Graf aleatori generat amb els paràmetres del model A amb node arrel \textbf{si}}
	\end{figure}
\end{itemize}

\newpage

\section{Model B}
\textbf{\textit{Xarxa bayesiana naive (model únic), sent la variable de classe de vidre la variable independent (i pare de totes les altres). Per tant, el valor d'U en aquest cas haurà de ser 0.}}\\
\begin{itemize}
\item Ajustem els paràmetres del K2 per satisfer les següents condicions:
	\begin{itemize}
	\item \textbf{Type variable independent i pare de totes les altres}: \textit{initAsNaiveBayes} $\rightarrow$ \textit{True} i \textit{(Nom)} $\rightarrow$ \textit{Type}
	\item \textbf{Màxim nombre de pares per variable}:  
 \textit{maxNrOfParents} $\rightarrow$ \textit{0}
	\end{itemize}
	\begin{figure}[hbtp]
	\centering
	\includegraphics[scale=0.4]{Figures/6.png}
	\caption{Paràmetres K2 model B}
	\end{figure}

\item Visualitzem el graf generat en l'execució:
	\begin{figure}[hbtp]
	\centering
	\includegraphics[scale=0.4]{Figures/r2.png}
	\caption{Graf aleatori generat amb els paràmetres del model B amb node arrel \textbf{type}}
	\end{figure}
\end{itemize}

\newpage

\section{Model C}
\textbf{\textit{Xarxes bayesianes obtingudes amb K2 a partir d'un model inicial que sigui la xarxa bayesiana naive del punt 2, però podent afegir arestes addicionals (i per tant U haurà de ser més gran que 0)}}
\begin{itemize}
\item Ajustem els paràmetres del K2 per satisfer les següents condicions:
	\begin{itemize}
	\item \textbf{Type variable independent i pare de totes les altres}: \textit{initAsNaiveBayes} $\rightarrow$ \textit{True} i \textit{(Nom)} $\rightarrow$ \textit{Type}
	\item \textbf{Màxim nombre de pares per variable}:  
 \textit{maxNrOfParents} $\rightarrow$ \textit{N, on N $\epsilon$  $\mathbb{N}$ }\\
	\end{itemize}
	\begin{figure}[hbtp]
	\centering
	\includegraphics[scale=0.4]{Figures/4.png}
	\caption{Paràmetres K2 model C}
	\end{figure}
\item Visualitzem el graf generat en l'execució:\\
	\begin{figure}[hbtp]
	\centering
	\includegraphics[scale=0.4]{Figures/r3.png}
	\caption{Graf aleatori generat amb els paràmetres del model C amb node arrel \textbf{type}}
	\end{figure}
\end{itemize}
\newpage

\part{Selecció del millor model}
\end{document}

