\documentclass[11pt,a4paper]{article}
\usepackage[utf8]{inputenc}
\usepackage[catalan]{babel}
\usepackage{amsmath}
\usepackage{amsfonts}
\usepackage{enumerate}
\usepackage{enumitem}
\usepackage{amssymb}
\usepackage{graphicx}
\usepackage{fancyhdr}
\usepackage{appendix}
\usepackage[hidelinks]{hyperref}
\usepackage{hyperref}
\usepackage{subfig}
\usepackage[ampersand]{easylist}
\usepackage{multirow}
\usepackage{amsmath}
\usepackage{amsfonts}
\newcommand\tab[1][1cm]{\hspace*{#1}}
\usepackage[left=2cm,right=2cm,top=2cm,bottom=2cm]{geometry} 
\renewcommand{\appendixname}{Webgrafia}
\renewcommand{\appendixtocname}{Webgrafia}
\renewcommand{\appendixpagename}{Webgrafia}


\begin{document}
\begin{titlepage}

\begin{flushleft}
Escola Politècnica Superior\\
\vspace*{0.15in}
Grau en Enginyeria Informàtica\\
\vspace*{0.15in}
Automatic Reasoning and Learning
\end{flushleft}

\begin{center}
\vspace{2.0cm}
\includegraphics[scale=0.3]{Figures/M-UdL.jpg} 
\vspace{2.0cm}

\begin{LARGE}
\textbf{Pràctica de Models Probabilístics:}\\ 
\vspace*{0.15in}
Indentificació de classes de vidres en l'escena del crim
\end{LARGE}
\vspace{1.0cm}

\begin{large}
\textbf{Data}: 25 de Juny, de 2018 \\
\textbf{Professor}: Ramón Béjar Torres \\
\begin{tabular}{ll}
\textbf{Nom: }Marc Melis Batalla & \textbf{DNI: }48257130W \\
\textbf{Nom: }Roger Truchero Visa  & \textbf{DNI: }48056539V \\
\end{tabular}
\end{large}

\vspace*{0.2in}
\begin{center}
\rule{120mm}{0.1mm}\\
\end{center}
\end{center}
\vspace*{0.2in}
\part*{Introducció}
L'objectiu del treball és realitzar aprenentatge basat en xarxes bayesianes, en lloc de basat en sistemes
de regles. Es tracta del problema de classificació de diferents classes de vidres en funció de diferents
propietats físiques dels mateixos. Aquest és el valor que té cada un dels 11 atributs que trobareu
per cada instància (propietats d'un cristall correctament identificat) en el fitxer glass.data que es troba juntament amb la pràctica.
\end{titlepage}

\lhead[\thepage]{\includegraphics[scale=0.05]{Figures/M-UdL.jpg}  }
\chead[]{\textbf{Pràctica de Models Probabilístics}}
\rhead[]{ARA}
\renewcommand{\headrulewidth}{0.5pt}
\renewcommand{\footrulewidth}{0.5pt}
\fancypagestyle{plain}{
\fancyhead[L]{}
\fancyhead[C]{}
\fancyhead[R]{\thepage}
\fancyfoot[L]{}
\fancyfoot[C]{}
\fancyfoot[R]{}
\renewcommand{\headrulewidth}{0pt}
\renewcommand{\footrulewidth}{0pt}
}
\pagestyle{fancy}
\vspace*{0.05in}

\tableofcontents

\vspace*{0.2in}

\listoffigures

\newpage


\end{document}

